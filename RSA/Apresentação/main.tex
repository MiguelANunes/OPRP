\documentclass[xcolor=table]{beamer}
\usepackage[utf8]{inputenc}
\usepackage[T1]{fontenc}
\usepackage[alf]{abntex2cite}	
\usepackage{udesc}
\usepackage{amsfonts,amsmath,amssymb,mathtools}
\usepackage{verbatim}
\usepackage{listings}
\usepackage[ddmmyyyy]{datetime}
\usepackage{hyperref, url}
\usepackage{graphicx}
\usepackage{multirow}
\usepackage{changepage}

\usepackage{pgfplots}
\usepackage{textcomp}
\usepackage{geometry}

\newcommand{\uglyphi}{\phi} % mantendo o \phi velho
\renewcommand \phi{\varphi}
\let \emptyset \varnothing

\graphicspath{{Figuras/}}
\setbeamertemplate{frametitle continuation}{}

% suprimindo warnings do hyperref
\pdfstringdefDisableCommands{%
  \def\\{}%
  \def\texttt#1{<#1>}%
  \def\smallskip{}%
  \def\medskip{}%
}

\lstset{language=C++,
    basicstyle=\ttfamily,
    keywordstyle=\color{blue}\ttfamily,
    stringstyle=\color{red}\ttfamily,
    commentstyle=\color{green}\ttfamily,
    morecomment=[l][\color{magenta}]{\#}
}

\renewcommand{\figurename}{Figura}
\sloppy
\title[]{Trabalho Final de Programação Paralela - Paralelizando Quebra de Chaves do Algoritmo RSA}

\author[Miguel Alfredo Nunes]{
    Miguel Alfredo Nunes\\\smallskip
    \texttt{miguel.alfredo.nunes@gmail.com}
}

\date{\today}

\begin{document}
    
    \begin{frame}
        \titlepage
    \end{frame}

    \begin{frame}[allowframebreaks]{Sumário}
        \tableofcontents
    \end{frame}

    \begin{frame}{Algoritmo RSA}

    \end{frame}
    
    \begin{frame}{Fatorando as Chaves}

    \end{frame}
    
    \begin{frame}{Detalhes da Implementação}
        
    \end{frame}

    \begin{frame}{Dificuldades na Implementação}

    \end{frame}

    \begin{frame}{Desempenho Sequecial}
   
    \end{frame}

    \begin{frame}{Desempenho Paralelo}
   
    \end{frame}

    \begin{frame}{Speed Up e Eficiência}
   
    \end{frame}

    \begin{frame}{Gráficos}

    \end{frame}

    % \begin{frame}{Implementação}
    %     \begin{figure}[htbp]
    %         \centering
    %         \includegraphics[scale=.6]{FramesModelosKTK4-3.png}
    %         \caption{Provas de Corretude das Funções}
    %     \end{figure}
    % \end{frame}

    % \begin{frame}{Implementação}
    %     \begin{adjustwidth}{-.6cm}{}
    %         \begin{figure}[htbp]
    %             \centering
    %             \includegraphics[scale=.5]{TraduçãoFórmulas.png}
    %             \caption{Traduzindo Fórmulas}
    %         \end{figure}
    %     \end{adjustwidth}
    % \end{frame}

    % \begin{frame}{Implementação}
    %     \begin{adjustwidth}{0cm}{}
    %         \begin{figure}[htbp]
    %             \centering
    %             \includegraphics[scale=.5]{GenralizaçãoValoração.png}
    %             \caption{Extensão da Valoração}
    %         \end{figure}
    %     \end{adjustwidth}
    % \end{frame}

    % \begin{frame}{Implementação}
    %     \begin{adjustwidth}{-.9cm}{}
    %         \begin{figure}[htbp]
    %             \centering
    %             \includegraphics[scale=.5]{SistemaDedutivo.png}
    %             \caption{Deduções em \(\textbf{KT} \odot \textbf{K4}\)}
    %         \end{figure}
    %     \end{adjustwidth}
    % \end{frame}
\end{document}


% \begin{adjustwidth}{-3cm}{}
%     \begin{tikzpicture}
%         \begin{axis}[
%             title={Tempo de Execução da Fatoração},
%             axis lines = left,
%             xlabel={Número de bits},
%             ylabel={Tempo de Execução [segundos]},
%             xmin=0, xmax=40,
%             ymin=0, ymax=400,
%             xtick={8,12,16,20,24,28,32,36,37},
%             ytick={0,50,100,200,300,390},
%             legend pos=north east,
%         ]
        
%         \addplot[
%             color=blue,
%             mark=square,
%             ]
%             coordinates {
%             (8,1.1765e-05)(12,6.522e-05)(16,0.00045097)(20,0.0130463)(24,0.25393)(28,1.91251)(32,42.8216)(36,100.659)(37,385.322)
%             };
            
%         \end{axis}
%     \end{tikzpicture}        
% \end{adjustwidth}